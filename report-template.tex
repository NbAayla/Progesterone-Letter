%%%%%%%%%%%%%%%%%%%%%%%%%%%%%%%%%%%%%%%%%
% kaobook
% LaTeX Template
% Version 1.3 (18/2/20)
%
% This template originates from:
% https://www.LaTeXTemplates.com
%
% For the latest template development version and to make contributions:
% https://github.com/fmarotta/kaobook
%
% Authors:
% Federico Marotta (federicomarotta@mail.com)
% Giuseppe Silano (g.silano89@gmail.com)
% Based on the doctoral thesis of Ken Arroyo Ohori (https://3d.bk.tudelft.nl/ken/en)
% and on the Tufte-LaTeX class.
% Modified for LaTeX Templates by Vel (vel@latextemplates.com)
%
% License:
% CC0 1.0 Universal (see included MANIFEST.md file)
%
%%%%%%%%%%%%%%%%%%%%%%%%%%%%%%%%%%%%%%%%%

%----------------------------------------------------------------------------------------
%	PACKAGES AND OTHER DOCUMENT CONFIGURATIONS
%----------------------------------------------------------------------------------------

\documentclass[
	%fontsize=10pt, % Base font size
	%twoside=false, % If true, use different layouts for even and odd pages (in particular, if twoside=true, the margin column will be always on the outside)
	secnumdepth=3, % How deep to number headings. Defaults to 2 (subsections)
	%abstract=true, % Uncomment to print the title of the abstract
	%numbers=noenddot, % Comment to output dots after section numbers; the most common values for this option are: enddot, noenddot and auto (see the KOMAScript documentation for an in-depth explanation)
	%draft=true, % If uncommented, rulers will be added in the header and footer
	%overfullrule=true, % If uncommented, overly long lines will be marked by a black box; useful for correcting spacing problems
]{kaohandt}

% Choose the language
\usepackage[english]{babel} % Load characters and hyphenation
\usepackage[english=british]{csquotes}	% English quotes

% Load packages for testing
\usepackage{blindtext}
%\usepackage{showframe} % Uncomment to show boxes around the text area, margin, header and footer
%\usepackage{showlabels} % Uncomment to output the content of \label commands to the document where they are used

%\graphicspath{{images/}{./}} % Paths where images are looked for

% Load mathematical packages for theorems and related environments. NOTE: choose only one between 'mdftheorems' and 'plaintheorems'.
% \usepackage{styles/mdftheorems}
\usepackage{styles/plaintheorems}

% Load the bibliography package
\usepackage{styles/kaobiblio}
\addbibresource{report-template.bib} % Bibliography file

% Load the package for hyperreferences
\usepackage{styles/kaorefs}

% Make LaTeX produce the files required to compile the glossary
%\makeglossaries

% Make LaTeX produce the files required to compile the nomenclature
%\makenomenclature 

\newcommand\tsub[1]{\textsubscript{#1}}
\newcommand\tsup[1]{\textsuperscript{#1}}

%----------------------------------------------------------------------------------------

\begin{document}

%----------------------------------------------------------------------------------------
%	BOOK INFORMATION
%----------------------------------------------------------------------------------------

\title[Template for a Kao Report (or Handout)]{Template for a Kao\\ Report (or Handout)}

\author[MF, JMC]{Michael Faraday\thanks{Royal Society of London} \and John McClane \thanks{New York City Police Department}}

\date{\today}

%----------------------------------------------------------------------------------------
%	TITLE AND ABSTRACT
%----------------------------------------------------------------------------------------

% \maketitle

% \margintoc

\begin{abstract}
\noindent [Doctor Name],\\\\
I am sending you this letter to ask for a prescription for bioidentical progesterone to aid my transition from male to female. Progesterone is a critical hormone for my affirmed sex. Without it, my ability to transition completely is compromised and I am at a heightened risk for severe health problems such as low bone mineral density, cardiovascular diseases, and breast cancer.
\end{abstract}

% {\noindent\textbf{Keywords:} \LaTeX, Kao, handout, article, report}

\medskip

%----------------------------------------------------------------------------------------
%	MAIN BODY
%----------------------------------------------------------------------------------------

\section{What Is Progesterone?}

In cisgendered women\sidenote{Meaning their gender identity as a woman aligns with their female sex assigned at birth}, progesterone (P\tsub{4}) is a sex hormone important for pregnancy, embryogenisis, and the regulation of ovulation and menstruation. However, P\tsub{4} also has important roles outside of the pelvic and sex organs. P\tsub{4} is critical for:

\subsection{Breasts}

\subsubsection{Lobuloalveolar Development}
When combined with the protein Prolactin, P\tsub{4} mediates lobuloalveolar maturation of the mammary glands, allowing for milk production, lactation, and breastfeeding\sidecite{macias_hinck_2012}.

\subsubsection{Ductal Development}
P\tsub{4} is necessary for the development of ductal branching within the breast and general maturation, including the enlargement of the areola\sidecite{robinson_2000}. Without it, as frequently seen in transgender women, breasts remain in Tanner stage 3 indefinitely\sidecite{fisher_castellini_2017}. This lack of breast maturation is why the majority of current transgender women seek breast augmentation surgery\sidecite{wierckx_mueller_weyers_caenegem_roef_heylens_tsjoen_2012}.

\subsubsection{Reduced Breast Cancer Risk}
Studies indicate that hormone replacement therapy with estrogen alone (what I am currently prescribed) results in a significant (1.29-fold) increase in the risk for invasive breast cancer. This risk elevation was eliminated with the inclusion of bioidentical progesterone.\\

\noindent While synthetic progestins\sidenote{Such as medroxyprogesterone acetate} have been found to increase the risk of breast cancer. In a study\sidecite{fournier_berrino_clavel-chapelon_2007} of postmenopausal women (n=80,377) from 1990 to 2002, compared to never using hormone replacement therapy:
\begin{itemize}
    \item HRT with estrogen alone lead to a 1.29-fold increased risk of invasive breast cancer.
    \item HRT with estrogen combined with synthetic estrogens led to a 1.69-fold increased risk of invasive breast cancer.
    \item HRT with estrogen combined with bioidentical progesterone lead to a relative risk of 1.00, meaning no change in risk from the control group that didn't use HRT.
\end{itemize}

\section{Skin Health}
Estrogen and P\tsub{4} receptors have been detected in the skin. During and after menopause, decreased levels of female sex hormones result in atrophy, thinning of the skin as well as a reduction is elasticity, firmness, and strength\sidecite{Holzer2005EffectsAS}.

% \begin{itemize}
%     \item Skin health.
%     \item Sexual function.
%     \item Nervous system functions.
%     \item Brain development.
%     \item Bone strength and mineral density.
%     \item Regulating gall-bladder activity.
%     \item Gum health.
%     \item Signalling insulin release (and other pancreatic functions).
%     \item Preventing endometrial cancer.
%     \item etc.
% \end{itemize}

%----------------------------------------------------------------------------------------
%	BIBLIOGRAPHY
%----------------------------------------------------------------------------------------

% The bibliography needs to be compiled with biber using your LaTeX editor, or on the command line with 'biber main' from the template directory

\printbibliography[title=Bibliography] % Set the title of the bibliography and print the references

%----------------------------------------------------------------------------------------
%	NOMENCLATURE
%----------------------------------------------------------------------------------------
% 
% % The nomenclature needs to be compiled on the command line with 'makeindex main.nlo -s nomencl.ist -o main.nls' from the template directory
% 
% \nomenclature{$c$}{Speed of light in a vacuum inertial frame}
% \nomenclature{$h$}{Planck constant}
% 
% \renewcommand{\nomname}{Notation} % Rename the default 'Nomenclature'
% \renewcommand{\nompreamble}{The next list describes several symbols that will be later used within the body of the document.} % Prepend this text to the nomenclature
% 
% \printnomenclature % Output the nomenclature

%----------------------------------------------------------------------------------------
%	GLOSSARY
%----------------------------------------------------------------------------------------
% 
% % The glossary needs to be compiled on the command line with 'makeglossaries main' from the template directory
% 
% \newglossaryentry{computer}{
% 	name=computer,
% 	description={is a programmable machine that receives input, stores and manipulates data, and provides output in a useful format}
% }
% 
% % Glossary entries (used in text with e.g. \acrfull{fpsLabel} or \acrshort{fpsLabel})
% \newacronym[longplural={Frames per Second}]{fpsLabel}{FPS}{Frame per Second}
% \newacronym[longplural={Tables of Contents}]{tocLabel}{TOC}{Table of Contents}
% 
% \setglossarystyle{listgroup} % Set the style of the glossary (see https://en.wikibooks.org/wiki/LaTeX/Glossary for a reference)
% \printglossary[title=Special Terms, toctitle=List of Terms] % Output the glossary, 'title' is the chapter heading for the glossary, toctitle is the table of contents heading

%----------------------------------------------------------------------------------------

\end{document}
